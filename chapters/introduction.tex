Induction Motor is a singly-excited device that works on the principle of Mutual Induction. It is similar to a transformer which also works on the same principle. The difference between a transformer and IM is the transformer is a static device, whereas the IM is a rotating device. In industry, 90% of use the induction motor because of necessary characteristics. It is an inherent self-start motor, and it does not require a permanent magnet, No brushes, No commutator rings, No position sensor. The induction motor also has a simple and robust operation, maintains a good power factor, less maintenance, is highly efficient, small in size, reliable, and cheaper than other types of motor.
This project deals with the hardware for monitoring the continuous parameters and speed control part of the Induction Motor. With the help of sensors, monitored parameters are a voltage sensor, current sensor, speed sensor.

\section{Objectives}
\begin{itemize}
    \item {For safe and economic data communication in industry or any other field, monitoring and controlling the operation of an induction motor depend on the internet of Things (IoT) is to do.}
    \item {By early fault detection, process interruption of the motor can be reduced, reducing damages of the motor in an industrial process to a more significant extent, which makes the motor more reliable.}
    \item {To protect the motor from overloading, over-current, and high temperature.}
\end{itemize}

\section{Faults}
In Induction Motor number of types of faults that occur widely, it is subdivided into three most important parts such as 
\begin{itemize}
    \item {Electrical faults: In electrical fault commonly occurs a single phasing fault, Reverse phase sequencing fault, oversupply voltage, overload fault, Earth fault, etc.}
    \item {Mechanical Faults: In mechanical fault usually occurs a rotor broken bar fault, stator and rotor winding defect, Bearing fault, etc.}
    \item {Environment Faults: In an environment, fault typically occurs the vibration of the motor, Induction Motor surrounding environment affects the performance of an Induction Motor such as moisture, temperature, etc.}
\end{itemize}

\section{Brief Importance}
Dynamic motor testing is quickly becoming a crucial part of all good predictive maintenance programs as innovations continue to evolve in software and hardware upgrades. Monitoring supports the reliability engineer's approach to maintaining a safe, effective, and productive operation. Monitoring motor performance with adequately trained technicians using modern equipment allows plant managers to dictate their downtime, improve plant operations, and quickly identify poorly performing equipment. Monitoring motors' performance and making necessary adjustments will improve reliability, extend the motor's life, and reduce the facility's overall operating cost.
This project is about IoT-based Induction Motor monitoring parameters such as voltage, current, speed based on the sensor. The speed of the induction motor is controlled with the help of the PWM technique, as its speed can be controlled easily by controlling the input power of frequency. Continuous monitoring of the parameters maintains the continuity of production in industries, making the motor reliable, i.e., the production of an industry can be increased. Also, it prevents any abnormality in the induction motor and detects the early fault in the induction motor. 
Suppose there is any fault that takes place in the motor. In that case, it will be determined by a sensor that senses the parameter values of voltage, current, speed, and the respective output values by the sensor give a signal to Arduino Uno, which sends a command to the computer that the motor should automatically be disconnected from the system.
